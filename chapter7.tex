%!TEX root = Thesis_main.tex

\chapter{Results}
\label{chapter7}

DObbiamo ricordarci anche di fare delle considerazioni sul passaggio tra una funzione di costo (fuori grasp. area) e la F. di costo dentro grasping area.

\section{Simulations results}
Considerazioni iniziali sulle traiettorie provate e perchè abbiamo scelto tali traiettorie (farei step e traiettoria grasping)
	\subsection{effect of N}
		
	\subsection{effect of different parameterization}
		
	\subsection{Movement of the system toward grasping area}

		\subsubsection{errors in tracking}
			
		\subsubsection{time elapsed}
			comparison con clasical MPC logic
		\subsubsection{J function components}

	\subsection{Trajectory tracking in grasping operation}

		\subsubsection{errors in tracking}
			
		\subsubsection{time elapsed}
			comparison con clasical MPC logic
		\subsubsection{J function components}
	
\section{Physical experiments results}
	
	\subsection{Movement of the system toward grasping area}

		\subsubsection{errors in tracking}
			
		\subsubsection{time elapsed}
			comparison con clasical MPC logic

	\subsection{Trajectory tracking in grasping operation}

		\subsubsection{errors in tracking}
			
		\subsubsection{time elapsed}
			comparison con clasical MPC logic
		\subsubsection{J function components}
