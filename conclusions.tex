%!TEX root = Thesis_main.tex

\chapter{Conclusions}
\label{conclusions}

The developed high level controller for the presented Mobile Manipulator, together with results from simulation and physical experiments, requires some considerations on the effectiveness of the presented approach. As shown in Chapter \ref{chapter7} our approach gives improvements in reducing solving time with respect to a standard approach with a nonlinear MPC with piecewise control action and terminal constraint. In particular, the dependence of the computational effort on $N$ is reduced. Anyway in Chapter \ref{chapter7} also drawbacks of this approach are presented like the dependence on the number of parameters and the affect of the increasing weight profile for the cost function. The overall results show an approach that, for many application, can represent a step towards online developing of nonlinear MPC for fast application. The usage of a nonlinear model based controller, as mentioned before, allows to represent and control complex systems taking into account many degrees of freadom and coupling effects but introduces further limitations. Nonlinearities in the model lead to a non-convex problem, local minima existence, higher computational effort and difficulties in assessing stability. The advantage of a nonlinear model instead of using linearization techniques is then a key question given that very good results can be nowdays acheived linearizing the model of the system obtaining a faster controller with all the benefit that follow. Nevertheless, the complexity of systems as well as the increasing requirements in performing complex task requires exact models and this implies high nonlinearities. Furthermore, the increasing computational capability of control units as well as research works on new techniques to treat optimization problems is moving towards solving easier nonlinearities issues. 


future: bisgonerebbe capire quale è il vanytaggio di un sistema non lineare rispetto a uno linearizzato