%!TEX root = Thesis_main.tex

\chapter{Conclusions}
\label{conclusions}

The developed high level controller for the presented Mobile Manipulator, together with results from simulation and physical experiments, requires some considerations on the effectiveness of the presented approach. As shown in Chapter \ref{chapter7} our approach gives improvements in reducing solving time with respect to a standard nonlinear MPC with piecewise control action and terminal constraint. In particular, the dependence of the computational effort on $N$ is reduced. Anyway in Chapter \ref{chapter7} also drawbacks of this approach are presented like the dependence on the number of parameters and the affect of the increasing weight profile for the cost function. The overall results show an approach that, for many application, can represent a step towards online developing of nonlinear MPC for fast application. The usage of a nonlinear model based controller, as mentioned before, allows to represent and control complex systems taking into account many degrees of freedom and coupling effects between them but introduces further limitations. Nonlinearities in the model lead to a non-convex problem, local minima existence, higher computational effort and difficulties in assessing stability. The advantage of a nonlinear model instead of using linearization techniques is then a key question, in fact very good results can be nowdays acheived linearizing the model of the system obtaining a faster controller with all the benefit that follow. Nevertheless, the complexity of systems as well as the increasing requirements in performing complex task requires exact models and this implies high nonlinearities. Furthermore, the increasing computational capability of control units as well as research works on new techniques to treat optimization problems is moving towards solving nonlinearities issues. Summarizing, the high level controller developed based on the presented approach gives it's contribution trying to reduce the computational effort towards fast NMPC application but to properly understand its capabilities future works are needed.

\section{Future Works}

- dire del codice che si può implementare meglio \\
- dire che si possono provare a verificare dove e come avvengono minimi locali \\
- dire che si possono fare delle prove con della sensoristica migliore \\
- dire che si può introdurre grasping fatto meglio studiando le traiettorie corrette \\
- dire che se il solutore si può settare meglio (tolleranze e componenti del solutore commerciale) \\
- dire che si può provare ad applicare questo approccio a linear MPC per ridurre la dipendenza sull'N \\
