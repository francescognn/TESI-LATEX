%!TEX root = Thesis_main.tex

\chapter{Conclusions}
\label{conclusions}
A nonlinear Model Predictive Control for trajectory tracking tasks has been developed for the presented Mobile Manipulator. As shown in Chapter \ref{chapter7} our approach gives improvements in reducing solving time with respect to a standard nonlinear MPC with piecewise constant control action and terminal constraints. In particular, the dependence of the computational effort on $N$ is reduced. Anyway in Chapter \ref{chapter7} also the limitations of this approach are presented, like the effect on the tracking accuracy of the parameterization of the control variables with a not sufficiently high order, and of the increasing weight profile of the cost function. The overall results show an approach that, for many applications, can represent a step towards the online implementation of nonlinear MPC. \\
The usage of a nonlinear model-based controller, as mentioned before, allows to represent and control complex systems taking into account many degrees of freedom and coupling effects between them. Nonetheless nonlinearities in the model lead to a non-convex problem, local minima existence, higher computational effort and difficulties in assessing stability. The advantage of a nonlinear model instead of using linearization techniques is then a key question; indeed, very good results can be nowadays achieved linearizing the model of the system obtaining a faster controller with all the benefit that follow. Nevertheless, the complexity of systems, as well as the increasing requirements in performing complex tasks, requires exact models and this implies high nonlinearities. Furthermore, the increasing computational capability of control units as well as research works on new techniques to treat optimization problems is moving towards solving nonlinearity issues.\\
Summarizing, the high-level controller developed based on the presented approach gives its contribution trying to reduce the computational effort towards fast NMPC applications, but to properly understand its capabilities future works are needed.

\section{Future Works}

The presented work proposes an approach towards fast nonlinear Model Predictive Control for online application, but leaves a consistent number of aspects that need to be further studied. Moreover, the presented approach can be tested with a linearized system to understand if the same benefits are present in linear MPC approaches, i.e. if the dipendece on $N$ is reduced.\\
Indeed, the developed implementation of this approach can be improved with some modifications. First of all, a stand alone C++ code to be run in ROS should be used in order to get a even faster controller.\\
Futhermore, the presence of local minima in the cost function minimization can be studied and a better understanding of the effect of different terms in the cost function is required for this. \\
The capabilities of the available sensors for the positioning system of the base affect significantly the position of the end effector and so the accuracy of the entire system. Because of that, the performance in real applications can be improved developing a better estimation system using filtering techniques. \\
Grasping, as presented in this work, can be improved studing different approaches that can be integrated with the controller, though giving optimized pre-planned trajectories can be a sufficient solution. \\
The setup of the used solver (IPOPT) can be analyzed and customized to further reduce the solving times. \\
Finally stability of this approach has to be further studied to define a mathematical demostration for it; in particular the study should investigate the role of exponent $b$ on the stability.


% - dire del codice che si può implementare meglio \\
% - dire che si possono provare a verificare dove e come avvengono minimi locali \\
% - dire che si possono fare delle prove con della sensoristica migliore \\
% - dire che si può introdurre grasping fatto meglio studiando le traiettorie corrette \\
% - dire che se il solutore si può settare meglio (tolleranze e componenti del solutore commerciale) \\
% - dire che si può provare ad applicare questo approccio a linear MPC per ridurre la dipendenza sull'N \\
