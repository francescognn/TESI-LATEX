%!TEX root = Thesis_main.tex
	\newpage
\chapter*{Abstract}

\addcontentsline{toc}{chapter}{Abstract}
Industry is increasingly going towards the complete automatization of processes, using, among others, mobile manipulators for transporting parts between machines and work stations in a time-optimal manner and therefore increasing the efficiency of the plant. 
USC (University of Southern California) has proposed in \cite{shantanuthakar} an algorithm for the definition of optimal trajectories for mobile manipulator pick-and-transport operations: the robot arm picks up the part during the motion of the mobile base, avoiding the latter to stop to allow the operation.
Once done the planning of the motion of the mobile manipulator, the controller able to perform such trajectory has to be designed. In our work we aimed to design a controller that could solve the mobile manipulation problem to drive properly the degrees of freedom of the whole system during its motion dealing with unexpected obstacles and unprecise positioning during grasping operation.\\
The control logic chosen for this task is Model Predictive Control. The difficulty of its implementation lies in the high non-linearity of the system coupled with the high level of redundancy in executing trajectory following tasks. In order to decrease the computational effort of the optimizer that rise from such a complexity, the input profile to be computed has been parametrized. In this way, the dependence of the computational time on the control horizon is reduced and only some parameters have to be found. Moreover to further simplify the definition of the problem, a terminal constraint-free approach has been introduced. To assess the stability of the controller introducing these modification, an extension of what done in \cite{alamir2018stability} is proposed.The controller has been designed in Matlab/Simulink and tested both in simulation and on the real system using ROS interface. Results and performances for different trajectories are presented and future works using this approach have been introduced at the end.  

\vspace{0.5cm}
\noindent 
