	\newpage
\chapter*{Abstract}

\addcontentsline{toc}{chapter}{Abstract}

Industry is increasingly going towards the complete automatization of processes, using, among others, mobile manipulators for transporting parts between machines and work stations in a time-optimal manner and therefore increasing the efficiency of the plant. 
USC (University of Southern California) has proposed in \cite{shantanuthakar} an algorithm for the definition of optimal trajectories for mobile manipulator pick-and-transport operations: the robot arm picks up the part during the motion of the mobile base, avoiding the latter to stop to allow the operation.
Once done the planning of the motion of the mobile manipulator, USC asked us to face the problem of its control.\\
In our work we aimed to a control which could drive properly all the degrees of freedom of the robot during its motion coping also with unexpected obstacles and events and unprecise positioning during grasping operation.\\
The control logic we chose for this task is Model Predictive Control. The difficulty of its implementation lies in the high non-linearity of the system, modelling it both kinematically or dynamically coping with high level of redundancy of the whole system in executing trajectory following tasks.\\
In order to decrease the computational effort of the optimizer the input profile to be computed is been parametrized. In this way the controller doesn’t have to find the optimal inputs for all the timesteps in the control horizon, but only some parameters to let the input follow a certain curve. \\
Moreover to further simplify the definition of the problem avoiding the definition of the terminal constraints and terminal cost of the MPC problem, which where necessary for the stability of the control, we used a monotonically increasing weighting profile of the running costs in the cost function, following the idea of \cite{alamir2018stability}.\\
The controller has been tested on the robot sending command messages to ROS through Simulink nodes. This way the controller could have been written in Matlab, using the Matlab-Interface of IPOPT as optimizer.
It has been seen that this controller ………….\\
It is useful for ……\\
Next steps should be ……..\\


\vspace{0.5cm}
\noindent 
