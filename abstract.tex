%!TEX root = Thesis_main.tex
	\newpage
\chapter*{Abstract}

\addcontentsline{toc}{chapter}{Abstract}
Mobile Manipulators are increasingly used in industry going towards the complete automatization of processes in a time-optimal manner and therefore increasing the efficiency of the plant. University of Southern California (USC) has proposed in \cite{shantanuthakar} an algorithm for the definition of optimal trajectories for Mobile Manipulator pick-and-transport operations. Once planned the motion of the Mobile Manipulator, the controller able to perform such trajectory has to be designed. \\
In our work we aimed to design a controller that could solve the mobile manipulation problem driving properly the degrees of freedom of the whole system during its motion dealing with unexpected obstacles and unprecise positioning during grasping operation.\\
The control logic chosen for this task is Model Predictive Control. The difficulty of its implementation lies in the high non-linearity of the system coupled with the high level of redundancy in the execution of trajectory following tasks. The presented work aims to design a controller based on the nonlinear model of the system instead of using linearizing techniques. In order to decrease the computational effort of the optimizer that rise from such complexity, a parameterized control input profile has been considered, drastically reducing the computational dependence on the control horizon length. Moreover, to further simplify the definition of the problem, a terminal constraint-free approach has been introduced as an extension of what done in \cite{alamir2018stability}. \\
The controller has been designed and simulated in Matlab/Simulink and subsequently tested on the physical system available at USC using ROS interface. Results and performances for different trajectories are presented.  

\vspace{0.5cm}
\noindent 
